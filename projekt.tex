%============================================================================
%----------------------------------------------------------------------------
% zpracování: make, make pdf, make desky, make clean
% připomínky posílejte na e-mail: bidlom AT fit.vutbr.cz
% vim: set syntax=tex
%============================================================================
\documentclass[english,cover]{fitthesis} % odevzdani do wisu - odkazy, na ktere se da klikat
%\documentclass[cover,print]{fitthesis} % pro tisk - na odkazy se neda klikat
%\documentclass[english,print]{fitthesis} % pro tisk - na odkazy se neda klikat
%      \documentclass[english]{fitthesis}
% * Je-li prace psana v anglickem jazyce, je zapotrebi u tridy pouzit 
%   parametr english nasledovne:
%      \documentclass[english]{fitthesis}
% * Neprejete-li si vysazet na prvni strane dokumentu desky, zruste 
%   parametr cover

% zde zvolime kodovani, ve kterem je napsan text prace
% "latin2" pro iso8859-2 nebo "cp1250" pro windows-1250, "utf8" pro "utf-8"
%\usepackage{ucs}
\usepackage[utf8]{inputenc}
\usepackage[T1, IL2]{fontenc}
\usepackage{url}
\usepackage[dvipsnames]{xcolor}
\usepackage{sectsty}
\usepackage{tikz}
\usetikzlibrary{decorations.pathmorphing}
\usepackage{graphicx}
\usepackage{subcaption}
\usepackage{amsmath}
\usepackage{amsthm}
\usepackage{comment}

\newtheorem{lemma}{Lemma}[section]

\newcommand{\vata}[0]{the VATA library}
\newcommand{\Vata}[0]{The VATA library}
\newcommand{\fagr}[0]{\otimes t_1, \ldots, t_n}
\newcommand{\funcdecl}[3]{#1: #2 \rightarrow #3}
\newcommand{\subst}[2]{[#1/#2]} %co, za co
\newcommand{\bexmp}[0]{\noindent\rule{\textwidth}{0.4pt} \begin{example}}
\newcommand{\eexmp}[0]{\noindent\rule{\textwidth}{0.4pt} \end{example}}


\DeclareUrlCommand\url{\def\UrlLeft{<}\def\UrlRight{>} \urlstyle{tt}}

%zde muzeme vlozit vlastni balicky


% =======================================================================
% balíček "hyperref" vytváří klikací odkazy v pdf, pokud tedy použijeme pdflatex
% problém je, že balíček hyperref musí být uveden jako poslední, takže nemůže
% být v šabloně
\ifWis
\ifx\pdfoutput\undefined % nejedeme pod pdflatexem
\else
  \usepackage{color}
  %\usepackage[unicode,colorlinks,hyperindex,plainpages=false,pdftex]{hyperref}
  \usepackage[unicode,hyperindex,plainpages=false,pdftex]{hyperref}
  \definecolor{links}{rgb}{0.4,0.5,0}
  \definecolor{anchors}{rgb}{1,0,0}
  \def\AnchorColor{anchors}
  \def\LinkColor{links}
  \def\pdfBorderAttrs{/Border [0 0 0] }  % bez okrajů kolem odkazů
  \pdfcompresslevel=9
\fi
\fi

%Informace o praci/projektu
%---------------------------------------------------------------------------
\projectinfo{
  %Prace
  project=DP,            %typ prace BP/SP/DP/DR
  year=2015,             %rok
  date=\today,           %datum odevzdani
  %Nazev prace
  title.cs={Verifikace ukazatelových programů pomocí lesních automatů},  %nazev prace v cestine
  title.en={Verification of Pointer Programs Based on Forest Automata}, %nazev prace v anglictine
  %Autor
  author={Martin Hruška},   %jmeno prijmeni autora
  %author.title.p=Bc., %titul pred jmenem (nepovinne)
  %author.title.a=PhD, %titul za jmenem (nepovinne)
  %Ustav
  department=UITS, % doplnte prislusnou zkratku: UPSY/UIFS/UITS/UPGM
  %Skolitel
  supervisor= Lukáš Holík, %jmeno prijmeni skolitele
  supervisor.title.p=Mgr.,   %titul pred jmenem (nepovinne)
  supervisor.title.a={Ph.D.},    %titul za jmenem (nepovinne)
  %Klicova slova, abstrakty, prohlaseni a podekovani je mozne definovat 
  %bud pomoci nasledujicich parametru nebo pomoci vyhrazenych maker (viz dale)
  %===========================================================================
  %Klicova slova
  keywords.cs={lesní automaty, formální verifikace, statická analýza, složité datové struktury, stromové automaty, zpětný běh, predikátová abstrakce.}, %klicova slova v ceskem jazyce
  keywords.en={forest automata, formal verification, static analysis, complex data structures, tree automata, backward run, predicate abstraction.}, %klicova slova v anglickem jazyce
  %Abstract
  abstract.cs={
	  Lesní automaty jsou formalismem používaným pro analýzu a verifikaci programu, které pracují s dynamickými datovými strukturami.
	  Koncept lesních automatů je založen na stromových automatech a verifikační procedura na tomto konceptu postavená byla implementována v nástroji Forester.
	  Ten implementuje vlastní knihovnu pro manipulaci se stromovými automaty.
	  Nicméně existuje knihovna VATA, která obsahuje vysoce efektivní implementaci algoritmů
	  pro práci se stromovými automaty a to především algoritmů pro test inkluze jazyků stromových automatů,
	  což je stěžejní operace i při práci s lesními automaty.
	  Prvním cílem této práce je tedy vytvoření varianty nástroje Forester založené na knihovně VATA.
	  Druhým cílem je potom rozšíření verifikační procedury založené na lesních automatů o zpětný běh, který
	  při nalezení chyby ve verifikovaném programu slouží k určení toho, zdali je chyba skutečná nebo byla způsobena
	  abstrakcí použitou během průběhu verifikace, což je informace použitelná pro zjemnění predikátové abstrakce.
	  První cíl byl již splněn a varianta nástroje Forester používající knihovnu VATA se účastnila soutěže SV-COMP 2015.
  }, % abstrakt v ceskem jazyce
  abstract.en={
	  Forest automata are formalism used for analysis and verification of programs manipulating dynamic data structures.
	  Forest automata are based on tree automata and shape analysis related to forest automata has been implemented in Forester tool.
	  Forester has its own implementation of tree automata.
	  However, there is the VATA library which implements the efficient algorithms for the tree automata manipulation,
	  especially the efficient algorithms for the checking language inclusion of tree automata what is operation
	  crucial also for verification procedure based on forest automata.
	  The first goal of this thesis is to implement version of Forester tool that uses VATA library for tree automata manipulation.
	  The second goal of this thesis is to extend forest automata based verification with backward run that checks whether
	  a found error is a spurious or real one what could be used for refinement of predicate abstraction.
	  The first goal has been already fulfill and the variant of Forester using the VATA library participated in the competition SV-COMP 2015.}, % abstrakt v anglickem jazyce
  %Prohlaseni
  declaration={Prohlašuji, že jsem tuto diplomovou práci vypracoval samostatně pod vedením\\ pana Mgr. Lukáše Holíka, Ph.D.},
  %Podekovani (nepovinne)
  acknowledgment={Rád bych poděkoval vedoucímu této práce, Mgr. Lukáši Holíkovi, Ph.D., za odborné rady a vedení při tvorbě této práce.
  Dále bych rád poděkoval Ing. Ondřeji Lengálovi za četné konzultace, během kterých mě trpělivě seznamoval s nástrojem Forester, a Prof. Ing. Tomáši Vojnarovi, Ph.D.
  za poskytnuté rady během konzultací.} % nepovinne
}

%Abstrakt (cesky, anglicky)
%\abstract[cs]{Do tohoto odstavce bude zapsán výtah (abstrakt) práce v českém jazyce.}
%\abstract[en]{Do tohoto odstavce bude zapsán výtah (abstrakt) práce v anglickém jazyce.}

%Klicova slova (cesky, anglicky)
%\keywords[cs]{Sem budou zapsána jednotlivá klíčová slova v českém jazyce, oddělená čárkami.}
%\keywords[en]{Sem budou zapsána jednotlivá klíčová slova v anglickém jazyce, oddělená čárkami.}

%Prohlaseni
%\declaration{Prohlašuji, že jsem tuto bakalářskou práci vypracoval samostatně pod vedením pana X...
%Další informace mi poskytli...
%Uvedl jsem všechny literární prameny a publikace, ze kterých jsem čerpal.}

%Podekovani (nepovinne)
%\acknowledgment{V této sekci je možno uvést poděkování vedoucímu práce a těm, kteří poskytli odbornou pomoc
%(externí zadavatel, konzultant, apod.).}

\begin{document}
  % Vysazeni titulnich stran
  % ----------------------------------------------
  \maketitle
  % Obsah
  % ----------------------------------------------
  \tableofcontents
  
  % Seznam obrazku a tabulek (pokud prace obsahuje velke mnozstvi obrazku, tak se to hodi)
  % \listoffigures
  % \listoftables 

  % Text prace
  % ----------------------------------------------
  \chapter{Introduction}
In last few decades importance of computers in our everyday lifes has laregly increased.
A lot of us can only hardly imagine doing their jobs without help of appropriate computer program
and we also spend a plenty of time using computers (e.g. personal computers or mobile devices) in leisure time.
There is also wide usage of computers in more critical applications like autopilot in an aeroplane
or programs for controlling power stations.
But growing number of applications of computer programs brings also need for their safety and security.

However, guarantee of software quality and correctness is not easy task
because programs has often many state which they go through during computation
and it could be very time and (memory) space demanding to check whether no undesirable thing
happens in any of the visited states.
One approach to ensure software qaulity is \emph{testing} (and dynamic analysis) which is basically based
on runnig a program in the different contexts and under the different inputs
and checking whether programs behaviour and outputs are expected one.
This method can satisfy many of requiremnts for software quality and often cover greate space of programs behaviours
but on the other side it is only possbile to prove presence of the errors using testing not their absence (\cite{djikstra}).
Moreover finding errors during testing does not mean that all of them has been found.

The mentioned weakness of testing can be resolved by \emph{formal verification}
which is another approach to checking program correctness.
Formal verification is method for checking whether a given system meets a given specification \cite{fav:lecture}.
There are three main branches of formal verification.
The first one is \emph{model checking} which systematically explores of a model (e.g. model of a program) to
prove that a property holds along the whole model.
The second approach is static analysis which is done over a source code (or some modification of it) of a system
without its explicit execution.
The last one is \emph{theorem proving}.
It proves the program in standard mathematical way -- starting from axioms and proving theorems to
verify the properties of a given system.
Theorem proving could be more or less automated.

This thesis deals with specific part of static analysis focused on verification of the programs manipulating
complex data structures (like a different kinds of lists and trees) allocated on the heap.
The properties checked for this class of programs are for example checking whether no dangling
pointers are dereferenced (no invalid dereference), whether all allocated memory on a heap is also freed (no memory leaks)
a program exuction or whether there is not dealocated pointer which has not have assigned memory (no invalid free).
There are different approaches to this kind of static analysis.
One approach is based on separation logic (some citations here) which has yet some successful implementation \cite{predator} which has been
evaluated on SV-Comp benchmark \cite{svcomp}.
Another approach is automata based, more precisly based on concept of \emph{Forest automata} and this one
will be main topic of this thesis.

The Forest automata has been introduced in \cite{forester} and implemented in tool called \emph{Forester} (appeared on SV-Comp 15).


\cite{Knuth}

formal verification

complex data structure and memory safety

forester and vata

outline

\chapter{Preliminaries}

\section{Graphs and Languages}
\section{Tree Automata}
\section{Forest Automata}

\chapter{VATA and Forester}

\section{VATA}

\section{Forester}

\section{Other Tools}

\chapter{??Predicate Abstraction and Backward Run??}

\chapter{Backward Run in Forest Automata Based Verification}

\section{Backward Run in Symbolic Context}

\section{Intersection of Forest Automata}

\chapter{Implementation}

\section{Execution Trace}
\section{Module For Intersection}

\chapter{Evaluation}

\chapter{Conclusion}
 % viz. obsah.tex

  % Pouzita literatura
  % ----------------------------------------------
\ifczech
  \bibliographystyle{czechiso}
\else 
  \bibliographystyle{plain}
%  \bibliographystyle{alpha}
\fi
  \begin{flushleft}
  \bibliography{literatura} % viz. literatura.bib
  \end{flushleft}
  \appendix
  
  \chapter{Storage Medium}
The storage medium contains the~source code of
the~Forester tool (that is distributed together with the~Predator tool
and Code Listener)
and the~source code of the~VATA library.
The tests used for the~evaluation in this thesis are
another content of the~storage medium.
It also contains an~electronic version of this technical report
and its \LaTeX\ sources.
The file \emph{README} describes the~structure
of the~storage medium and the~instructions for compiling
and running Forester.
%\chapter{Obsah CD}
%\chapter{Manual}
%\chapter{Konfigrační soubor}
%\chapter{RelaxNG Schéma konfiguračního soboru}
%\chapter{Plakat}

 % viz. prilohy.tex
\end{document}
